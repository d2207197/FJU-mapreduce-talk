
\documentclass[12pt,c]{beamer}
\usetheme[noflama]{nlplab}

\let\oldframe\frame
\renewcommand\frame[1][plain]{\oldframe[#1]}


\usepackage{listings}
\usepackage{minted}
\usepackage{tabu}

%======================================================================
% 中文字體 
%======================================================================
\usepackage{xeCJK}
\setCJKmainfont[AutoFakeBold=5,AutoFakeSlant=.4]{SimHei}
\setCJKsansfont{SimHei} \setCJKmonofont{SimSun}


%======================================================================
% 檔案資訊
%======================================================================
\title{Map Reduce}
% \subtitle{語言搜尋引擎}
% \institute{NLP Lab}
% \author{}
% \date{}

%----------------------------------------------------------------------
% 圖形路徑
%----------------------------------------------------------------------
\graphicspath{ {./images/} }


\lstset{basicstyle=\ttfamily\color{treelabWarmGreyDark}}
%======================================================================
% 檔案開始
%======================================================================
\begin{document}

\begin{frame}
  \titlepage
\end{frame}

\begin{frame}{巨量資料}
  \begin{itemize}
    \item 20+ billion web pages x 20KB = 400+ TB
    \item 單台電腦讀取硬碟速度 30-35 MB/sec。需 4 個月讀取整個 web
    \item 200 * 2TB 硬碟來儲存
    
  \end{itemize}
  \begin{itemize}
  \item Facebook 有 10 billion 張照片
  \item 需 petabytes 以上儲存空間
  \end{itemize}
\end{frame}

\begin{frame}{巨量資料}
  \begin{itemize}
    \item 分散式計算、分散式儲存 
    \item 2011 年時 Google 有 1M 台機器
      \item 
  \end{itemize}
    \includeimage{cluster-architecture}
\end{frame}

\begin{frame}{MapReduce}
\begin{itemize}
\item 挑戰
\begin{itemize}
\item 如何分散計算、分散資料?
\item 撰寫分散式/平行程式很困難?
\end{itemize}
\item MapReduce 解決上述問題
\begin{itemize}
  \item Google 所設計的運算與資料處理模型
  \item 簡單且優雅的運用在大資料處理,撰寫平行程式不再困難
\end{itemize}
\end{itemize}
  


\end{frame}

\begin{frame}{MapReduce}
  \includeimage{map-reduce}
\end{frame}

\begin{frame}[containsverbatim]{範例: Word Count}
\begin{minted}[mathescape,linenos,numbersep=5pt,frame=lines,framesep=2mm]{console}
$ cat data | tr -sc 'A-Za-z' '\n'  | sort  | uniq -c
...
    645 and
      2 animation
      3 annotations
      3 answers
      1 anticipated
     19 any
      2 anymore
      2 anything
     39 apos
...
\end{minted}
\end{frame}



\begin{frame}[containsverbatim]{範例: Word Count}
\begin{minted}[mathescape,linenos,numbersep=5pt,frame=lines,framesep=2mm]{console}
$ cat data 
...
On January 2, 1985, Zaman Akil sent the Academy of Sciences a short summary of a longer work.
At his request, the Perpetual Secretary of the Academy, Prof. Paul Germain, sent the letter to several members of the Academy, including myself.
I was the only one who agreed to discuss it with the author.
...
\end{minted}
\end{frame}

\begin{frame}{Map}
  \includeimage{map}
\end{frame}


\begin{frame}[containsverbatim]{範例: Word Count}
\begin{minted}[mathescape,linenos,numbersep=5pt,frame=lines,framesep=2mm]{console}
$ cat data | tr -sc 'A-Za-z' '\n' 
...
Germain
sent
the
letter
to
several
members
of
...
\end{minted}
\end{frame}

\begin{frame}{MapReduce}
  \includeimage{group}
\end{frame}



\begin{frame}[containsverbatim]{範例: Word Count}
\begin{minted}[mathescape,linenos,numbersep=5pt,frame=lines,framesep=2mm]{console}
$ cat data | tr -sc 'A-Za-z' '\n' | sort
...
Also
Also
Also
Although
Although
America
American
American
Among
An
An
...

\end{minted}
\end{frame}

\begin{frame}{MapReduce}
  \includeimage{reduce}
\end{frame}


\begin{frame}[containsverbatim]{範例: Word Count}
\begin{minted}[mathescape,linenos,numbersep=5pt,frame=lines,framesep=2mm]{console}
$ cat data | tr -sc 'A-Za-z' '\n'  | sort  | uniq -c
...
    645 and
      2 animation
      3 annotations
      3 answers
      1 anticipated
     19 any
      2 anymore
      2 anything
     39 apos
...
\end{minted}
\end{frame}



\begin{frame}[containsverbatim]{WordCount Functions}
\begin{minted}[mathescape,linenos,numbersep=3pt,frame=lines,framesep=2mm]{python}
def map(key, value):
    # key: NA; value: a line of input text
    for word in value:
        emit(word, 1)
def reduce(key, values):
    # key: word; values: an iterator over counts
    result = 0
    for count in values:
        result += count
    emit(key, result)
\end{minted}
\end{frame}

% \begin{frame}{MapReduce}
%   \includeimage{map-reduce-2}
% \end{frame}


\begin{frame}{目標}
  \begin{itemize}
  \item 建立一搜尋引擎用於搜尋英文詞語用法。
  \item 可輔助英語學習與文章寫作。
  \end{itemize}

  \begin{block}{搜尋例子}
    \begin{itemize}
    \item \lstinline|adj. beach|: 即代表搜尋 beach 前面出現過的形容詞。
    \item \lstinline|play * role|: 搜尋 play 與 role 中間最常出現的字詞組合。
    \item \lstinline|go ?to home|: go 與 home 之間是否要放 to。
    \item \lstinline|go * movie|: go 與 role 中間最常出現的字詞組合。 
    \item \lstinline|kill the _|: 最常被 kill 的東西是。
    \end{itemize}
  \end{block}
\end{frame}

\begin{frame}[t,plain]{用 Google 查英文}
  \includeimage{google-search-beach}
\end{frame}

\begin{frame}[t,plain]{用 Google 查英文}
  \includeimage{google-search-play-role}
\end{frame}


\begin{frame}[plain,shrink=5]{語法設計}
  \begin{tabu}{X[1.2]X[2]}
    \hline
    \rowfont{\bf} 語法 & 說明  \\
    \hline
    \lstinline/_/ & 單一任意字詞 \\
    \lstinline/*/ & 零到多個任意字詞  \\
    \lstinline/?term/ & term 可有可無 \\
    \lstinline!term1 | term2! & term1 或 term2 \\
    \lstinline/adj. det. n. v. prep./ & 形容詞、冠詞、名詞、動詞、介繫詞 \\
    \hline
  \end{tabu}

  \begin{block}{搜尋例子}
    \begin{itemize}
    \item \lstinline/adj. beach/: 即代表搜尋 beach 前面出現過的形容詞。
    \item \lstinline/play * role/: 搜尋 play 與 role 中間最常出現的字詞組合。
    \item \lstinline/go ?to home/: go 與 home 之間是否要放 to。
    \item \lstinline/go * movie/: go 與 role 中間最常出現的字詞組合。 
    \item \lstinline/kill the _/: 最常被 kill 的東西是。
    \end{itemize}
  \end{block}
\end{frame}

\begin{frame}{Lab 12}
  \begin{itemize}
  \item 目標:完成語法第一項 \lstinline/_/
    \begin{itemize}
    \item 任意位置置入 \lstinline/_/
    \item 最長 4-gram
    \end{itemize}
    \begin{block}{Query 範例}
      \begin{itemize}
      \item \lstinline/play _ _ role/
      \item \lstinline/kill the _/
      \item \lstinline/a _ beach/
      \end{itemize}
    \end{block}
  \item 輸入資料:citeseerx 的許多句子
  \item 輸出結果:
    \begin{itemize}
    \item key: 所有會有結果的 query
    \item value: 符合 query 的前 100 名 ngram 與 count。
    \end{itemize}
  \end{itemize}
%   \begin{block}{輸出範例}
%     \begin{tabu}{XXX}
%       \hline 
%       Key & Ngrams & Counts \\
%       \hline
%       a \_ beach & a sandy beach & 486 \\
%       & a private beach  & 416  \\
%       & a beautiful beach & 314  \\
%       & a small beach  & 175 \\
%       & ... \\
%       \hline
%     \end{tabu}
% \end{block}

\end{frame}
\begin{frame}[plain,shrink=5]{Lab 12 - 輸出}
  \begin{itemize}
  \item key: 所有會有結果的 query
  \item value: 符合 query 的前 100 名 ngram 與 count。
  \end{itemize}

  \begin{block}{輸出範例}
    \begin{tabu} to .95\textwidth {XX[2]X[1,r]}
      \hline 
      \rowfont{\bf}Key & Ngrams & Counts \\
      \hline
      \lstinline/a \_ beach/
      & a sandy beach & 486 \\
      & a private beach  & 416  \\
      & a beautiful beach & 314  \\
      & a small beach  & 175 \\
      & ... \\
      \tabucline[on 2pt]{-}

      \lstinline/kill the \_/
      & kill the people & 189 \\
      & kill the other & 174 \\
      & kill the process & 163 \\
      & kill the enemy & 160 \\
      & ... \\
      \hline
    \end{tabu}
  \end{block}
\end{frame}

\begin{frame}{隨堂測驗}
目標
  \begin{itemize}
  \item 依 MapReduce 架構,設計每階段 mapper, reduce 的輸入輸出來完成 Lab 12
  \item 在紙寫撰寫簡單輸入、輸出的 key-value 範例表達概念即可
  \end{itemize}
小提示
  \begin{itemize}
  \item 可有 1 至多個 map, reduce 流程
  \item 考慮 mapper 的輸入資料切割影響
  \item mapper 輸入為 value 或 key-value,輸出為 key-value
  \item reducer 輸入為 grouped key-values,輸出為 key-value
  \end{itemize}
\end{frame}

\begin{frame}[plain,shrink=40]{Bi-gram Count}
  \begin{block}{Bi-gram Count Mapper範例}
    \begin{tabu} to .95\textwidth {XX}
      \hline
      Input(\lstinline/value/) & Output(\lstinline/key => value/) \\
      \hline
      C D C D
      & C D => 2 \\
      & D C => 1 \\
      \tabucline[on 2pt]{-}
      B C D A 
      & B C => 1 \\
      & C D => 1 \\
      & D A => 1 \\
      \tabucline[on 2pt]{-}
      C D A B
      & C D => 1 \\
      & D A => 1 \\
      & A B => 1 \\
      \hline
    \end{tabu}
  \end{block}
  \begin{block}{Reducer 範例}
    \begin{tabu} to .95\textwidth {XX}
      \hline
      Input(\lstinline/key => value/) & Output(\lstinline/key => value/) \\
      \hline
      A B => 1 & A B => 1 \\
      \tabucline[on 2pt]{-}
      B C => 1 & B C => 1 \\
      \tabucline[on 2pt]{-}
      C D => 2 & C D => 4 \\
      C D => 1 & \\
      C D => 1 & \\
      \tabucline[on 2pt]{-} 
      D A => 1 & D A => 2 \\
      D A => 1 &  \\
      \tabucline[on 2pt]{-} 
      D C => 2 & C C => 2 \\
      \hline
    \end{tabu}
  \end{block}  
\end{frame}

\begin{frame}[plain,shrink=10]{Lab12}
  \begin{block}{Lab12 Mapper 範例}
    \begin{tabu} to .95\textwidth {XX}
      \hline
      Input(\lstinline/value/) & Output(\lstinline/key => value/) \\
      \hline
      \lstinline/A B C 200/
      & \lstinline/A B C => A B C 200/ \\
      & \lstinline/\_ B C => A B C 200/ \\
      & \lstinline/A \_ C => A B C 200/ \\
      & \lstinline/A B \_ => A B C 200/ \\
      & \lstinline/\_ \_ C => A B C 200/ \\
      & \lstinline/\_ B \_ => A B C 200/ \\
      & \lstinline/A \_ \_ => A B C 200/ \\
      \tabucline[on 2pt]{-}
      \lstinline/A D C 300/
      & \lstinline/\_ D C => A D C 300/ \\
      & \lstinline/A \_ C => A D C 300/ \\
      & ... \\
      \tabucline[on 2pt]{-}
      \lstinline/A E C 100/
      & \lstinline/\_ E C => A E C 100/ \\
      & \lstinline/A \_ C => A E C 100/ \\
      & ... \\
      \hline
    \end{tabu}
  \end{block}
\end{frame}

\begin{frame}[plain,shrink=10]{Lab12}
  \begin{block}{Lab12 Reducer 範例}
    \begin{tabu} to .95\textwidth {X[1.5]X[0.5]X}
      \hline
      Input(\lstinline/value/) & Output(\lstinline/key => value/) \\
      \hline
      \lstinline/A \_ C => A B C 200/ & \lstinline/A \_ C =>/ & \lstinline/A D C 300,/\\
      \lstinline/A \_ C => A D C 300/ & & \lstinline/A B C 200,/\\
      \lstinline/A \_ C => A E C 100/ & & \lstinline/A E C 100/\\
      \tabucline[on 2pt]{-}
      \lstinline/A B \_ => A B C 200/ & \lstinline/A B \_ =>/ & \lstinline/A B C 200/ \\
      \tabucline[on 2pt]{-}
      \lstinline/A D \_ => A D C 300/ & \lstinline/A D \_ =>/ & \lstinline/A D C 300/ \\
      \tabucline[on 2pt]{-}
      \lstinline/A E \_ => A E C 100/ & \lstinline/A E \_ =>/ & \lstinline/A E C 100/ \\
      \tabucline[on 2pt]{-}
      \lstinline/A \_ \_ => A B C 200/ & \lstinline/A \_ \_ =>/ & \lstinline/A D C 300,/ \\
      \lstinline/A \_ \_ => A D C 300/ & & \lstinline/A B C 200,/ \\
      \lstinline/A \_ \_ => A E C 100/ & & \lstinline/A E C 100/ \\
      \tabucline[on 2pt]{-}
      \lstinline/\_ B C => A B C 200/ & \lstinline/\_ B C =>/ & \lstinline/A B C 200/ \\
      \tabucline[on 2pt]{-}
      \lstinline/\_ D C => A D C 300/ & \lstinline/\_ D C =>/ & \lstinline/A D C 300/ \\
      \tabucline[on 2pt]{-}
      \lstinline/\_ E C => A E C 100/ & \lstinline/\_ E C =>/ & \lstinline/A E C 100/\\
      \tabucline[on 2pt]{-}
      ... & ...\\
      \hline
    \end{tabu}
  \end{block}
\end{frame}

\begin{frame}{Lab12}

需完成六支程式
\begin{itemize}
  \item 產生 ngram count 的 mapper, reducer
  \item 產生 query result 的 mapper, reducer
  \item 將 query result 轉為 database \\(試試 python 內建的 shelve 或 sqlite3 套件)
  \item Database 介面程式,讓使用者輸入 query ,即時取得 result
\end{itemize}  
\end{frame}

% \lstset{language=python}
\begin{frame}[containsverbatim]{python shelve}
\begin{minted}[mathescape,
               linenos,
               numbersep=5pt,
               % gobble=2,
               frame=lines,
               framesep=2mm]{python}

import shelve
d = shelve.open('data.shelve')
d['odds'] = [1, 3, 5, 7, 9]
print d['odds']
d['evens'] = [2, 4, 6, 8, 10]
d['hello'] = 'world'
del d['hello']
d['zipcodes'] = {'hsinchu': 300, 'zhongli': 320}
print d.keys()
d.close()

\end{minted}
Google ``python shelve'' for official documents
\end{frame}  


\begin{frame}[allowframebreaks,t]{Ngram Count}
\inputminted[mathescape,
               linenos,
               numbersep=5pt,
               fontsize=\scriptsize,
               frame=lines,
               framesep=2mm]{python}{ngramcount.py}

\end{frame}

\end{document}

%%% Local Variables:
%%% TeX-engine: xetex
%%% TeX-command-extra-options: "-shell-escape"
%%% End: