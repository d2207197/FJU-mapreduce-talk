\documentclass[12pt,c]{beamer}
\usetheme[noflama]{nlplab}

\let\oldframe\frame
\renewcommand\frame[1][plain]{\oldframe[#1]}


\usepackage{listings}
\usepackage{tabu}
%======================================================================
% 中文字體 
%======================================================================
\usepackage{xeCJK}
\setCJKmainfont[AutoFakeBold=5,AutoFakeSlant=.4]{SimHei}
\setCJKsansfont{SimHei} \setCJKmonofont{SimSun}


%======================================================================
% 檔案資訊
%======================================================================
\title{Unit 12}
\subtitle{語言搜尋引擎}
% \institute{NLP Lab}
% \author{}
% \date{}

%----------------------------------------------------------------------
% 圖形路徑
%----------------------------------------------------------------------
\graphicspath{ {./images/} }


\lstset{basicstyle=\ttfamily\color{treelabWarmGreyDark}}
%======================================================================
% 檔案開始
%======================================================================
\begin{document}

\begin{frame}
  \titlepage
\end{frame}

\begin{frame}{目標}
  \begin{itemize}
  \item 建立一搜尋引擎用於搜尋英文詞語用法。
  \item 可輔助英語學習與文章寫作。
  \end{itemize}

  \begin{block}{搜尋例子}
    \begin{itemize}
    \item \lstinline|adj. beach|: 即代表搜尋 beach 前面出現過的形容詞。
    \item \lstinline|play * role|: 搜尋 play 與 role 中間最常出現的字詞組合。
    \item \lstinline|go ?to home|: go 與 home 之間是否要放 to。
    \item \lstinline|go * movie|: go 與 role 中間最常出現的字詞組合。 
    \item \lstinline|kill the _|: 最常被 kill 的東西是。
    \end{itemize}
  \end{block}
\end{frame}

\begin{frame}[t,plain]{用 Google 查英文}
  \includeimage{google-search-beach}
\end{frame}

\begin{frame}[t,plain]{用 Google 查英文}
  \includeimage{google-search-play-role}
\end{frame}


\begin{frame}[plain,shrink=5]{語法設計}
  \begin{tabu}{X[1.2]X[2]}
    \hline
    \rowfont{\bf} 語法 & 說明  \\
    \hline
    \lstinline/_/ & 單一任意字詞 \\
    \lstinline/*/ & 零到多個任意字詞  \\
    \lstinline/?term/ & term 可有可無 \\
    \lstinline!term1 | term2! & term1 或 term2 \\
    \lstinline/adj. det. n. v. prep./ & 形容詞、冠詞、名詞、動詞、介繫詞 \\
    \hline
  \end{tabu}

  \begin{block}{搜尋例子}
    \begin{itemize}
    \item \lstinline/adj. beach/: 即代表搜尋 beach 前面出現過的形容詞。
    \item \lstinline/play * role/: 搜尋 play 與 role 中間最常出現的字詞組合。
    \item \lstinline/go ?to home/: go 與 home 之間是否要放 to。
    \item \lstinline/go * movie/: go 與 role 中間最常出現的字詞組合。 
    \item \lstinline/kill the _/: 最常被 kill 的東西是。
    \end{itemize}
  \end{block}
\end{frame}

\begin{frame}{Lab 12}
  \begin{itemize}
  \item 目標:完成語法第一項 \lstinline/_/
    \begin{itemize}
    \item 任意位置置入 \lstinline/_/
    \item 最長 4-gram
    \end{itemize}
    \begin{block}{Query 範例}
      \begin{itemize}
      \item \lstinline/play _ _ role/
      \item \lstinline/kill the _/
      \item \lstinline/a _ beach/
      \end{itemize}
    \end{block}
  \item 輸入資料:citeseerx 的許多句子
  \item 輸出結果:
    \begin{itemize}
    \item key: 所有會有結果的 query
    \item value: 符合 query 的前 100 名 ngram 與 count。
    \end{itemize}
  \end{itemize}
%   \begin{block}{輸出範例}
%     \begin{tabu}{XXX}
%       \hline 
%       Key & Ngrams & Counts \\
%       \hline
%       a \_ beach & a sandy beach & 486 \\
%       & a private beach  & 416  \\
%       & a beautiful beach & 314  \\
%       & a small beach  & 175 \\
%       & ... \\
%       \hline
%     \end{tabu}
% \end{block}

\end{frame}
\begin{frame}[plain,shrink=5]{Lab 12 - 輸出}
  \begin{itemize}
  \item key: 所有會有結果的 query
  \item value: 符合 query 的前 100 名 ngram 與 count。
  \end{itemize}

  \begin{block}{輸出範例}
    \begin{tabu} to .95\textwidth {XX[2]X[1,r]}
      \hline 
      \rowfont{\bf}Key & Ngrams & Counts \\
      \hline
      \lstinline/a \_ beach/
      & a sandy beach & 486 \\
      & a private beach  & 416  \\
      & a beautiful beach & 314  \\
      & a small beach  & 175 \\
      & ... \\
      \tabucline[on 2pt]{-}

      \lstinline/kill the \_/
      & kill the people & 189 \\
      & kill the other & 174 \\
      & kill the process & 163 \\
      & kill the enemy & 160 \\
      & ... \\
      \hline
    \end{tabu}
  \end{block}
\end{frame}

\begin{frame}{隨堂測驗}
目標
  \begin{itemize}
  \item 依 MapReduce 架構,設計每階段 mapper, reduce 的輸入輸出來完成 Lab 12
  \item 在紙寫撰寫簡單輸入、輸出的 key-value 範例表達概念即可
  \end{itemize}
小提示
  \begin{itemize}
  \item 可有 1 至多個 map, reduce 流程
  \item 考慮 mapper 的輸入資料切割影響
  \item mapper 輸入為 value 或 key-value,輸出為 key-value
  \item reducer 輸入為 grouped key-values,輸出為 key-value
  \end{itemize}
\end{frame}

\begin{frame}[plain,shrink]{Bi-gram Count}
  \begin{block}{Bi-gram Count Mapper範例}
    \begin{tabu} to .95\textwidth {XX}
      \hline
      Input(\lstinline/value/) & Output(\lstinline/key => value/) \\
      \hline
      C D C D
      & C D => 2 \\
      & D C => 1 \\
      \tabucline[on 2pt]{-}
      B C D A 
      & B C => 1 \\
      & C D => 1 \\
      & D A => 1 \\
      \tabucline[on 2pt]{-}
      C D A B
      & C D => 1 \\
      & D A => 1 \\
      & A B => 1 \\
      \hline
    \end{tabu}
  \end{block}
  \begin{block}{Reducer 範例}
    \begin{tabu} to .95\textwidth {XX}
      \hline
      Input(\lstinline/key => value/) & Output(\lstinline/key => value/) \\
      \hline
      A B => 1 & A B => 1 \\
      \tabucline[on 2pt]{-}
      B C => 1 & B C => 1 \\
      \tabucline[on 2pt]{-}
      C D => 2 & C D => 4 \\
      C D => 1 & \\
      C D => 1 & \\
      \tabucline[on 2pt]{-} 
      D A => 1 & D A => 2 \\
      D A => 1 &  \\
      \tabucline[on 2pt]{-} 
      D C => 2 & C C => 2 \\
      \hline
    \end{tabu}
  \end{block}  
\end{frame}

\end{document}